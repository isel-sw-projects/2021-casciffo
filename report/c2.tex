\chapter{}\label{ch:intro}
This chapter consists of the introduction of the agencies involved in
this application, as well as give a detailed view on the goals to be
achieved in this thesis.

\section{Introduction}\label{ch:intro:sec:intro}
The \href{https://hff.min-saude.pt/}{Hospital Professor Doutor Fernando Fonseca (HFF)}, is a first of the line hospital for the near 550.000 inhabitants of the municipalities of Amadora and Sintra. The Institution develops assistance and investigation activities as well as providing education, pre- and post-graduation training.

The Hospital's mission is to provide humanized and differentiated health
care throughout a person's life cycle, in collaboration with primary and
continuing health care, as well as other hospitals in the National
Health Service's (``Serviço Nacional de Saúde'') network.

The Clinical Investigation Unit (``Unidade de Investigação Clínica'',
UIC), created in March 2018, is an internal department within the
Hospital that incorporated fundamental concepts of activities in line
with the strategic objectives of the institution. The UIC is responsible
for managing clinical investigations and is characterized by a
multidisciplinary team responsible for ensuring accuracy in the
scientific planning of the studies submitted, for the fulfillment of
clinical best practices by researchers and for the negotiation of
contracts for projects financed by external promoters.\\
UIC's mission is to promote quality clinical investigation in an
organized and sustainable manner, following guidelines that value
systematic knowledge through the management of interfaces associated
with Investigation, Development, and Innovation, and complying with
applicable ethical and legal provisions, for the benefit of the
Hospital, the Community, the Patients, and the Families/Caregivers.\\
The goal is to become a reference of the promotion of best practices in
hospital clinical research and to consolidate a transversal scientific
culture within the institution.

Clinical investigations are very important to the world of medicine and
health care. It's through them that new medicine and new ways of
treatment are discovered and tested through strict adherence to 
scientific accuracy measurements and best practices before being
administrated to the public. \\

There are two main types of clinical investigations, without intervention, which includes observational trials (``Estudos Observacionais''), and with intervention, which includes clinical trials (``Ensaios Clínicos'').\\
Observational Trials consist purely of observation, for example in the
evaluation of a potential risk factor. On the other hand, Clinical
Investigations with intervention can be characterized as any
intervention that foresees any change, influence or programming of
health care, behavior or knowledge of the participating patients or
caretakers, with the end goal of discovering the effects it had on the
participant's health.\\
Clinical trials consist of a scientific controlled investigation, done
on humans (healthy or ill), with the end-goal of establishing or confirming the safety and efficiency of the experimental medicine.  

Clinical trials have shown to be a vital tool in the development and testing of vaccinations and treatments for the safety of the entire world population, especially now during the present Covid19 pandemic. For this reason, the efficiency and simplicity in managing and monitoring the evolution of a clinical trial is essential.  
The current procedure of a clinical investigation relies on email exchanges with both external and internal parties involved, which adds a considerable amount of effort in the management and monitoring of clinical trials. Another concern is with the scheduling and monitoring of Clinical Trials patients, as there is currently no systematic way of distinguishing the types of appointments made for each patient.  
In this context, the application detailed in this document, CASCIFFO, aims to provide a solution in order to enhance the efficiency in the management and monitoring of clinical investigations.


\section{CASCIFFO}\label{ch:intro:sec:casciffo}

CASCIFFO is a joint project between \href{https://hff.min-saude.pt/}{HFF} and \href{https://www.isel.pt/}{ISEL}. It concerns the application to the clinical investigation Agency Award and Biomedical Innovation (AICIB).  
CASCIFFO is a web-app that aligns with the UIC's goals by promoting efficiency and quality in the management of clinical research.
CASCIFFO strives to make the visualization, monitoring, and management of clinical Trials as simple and straightforward as possible.   
It will allow the UIC/HFF to be modernized, bringing a shift in how patients, researchers, and promoters view and value their institution.  

The application aims to develop and provide innovative mechanisms for interoperability with internal and external information systems, allowing, when desired, data synchronization, index search, identification data management and even access to detailed clinical data. 
CASCIFFO consists of two core modules, the front-end and the back-end. The front-end supports interaction with users while the back-end connects to an internal database system to the HFF/UIC and which aggregates the total information of this ecosystem.  
The interaction with users will depend on their role within the platform, displaying the appropriate information to each one. 


