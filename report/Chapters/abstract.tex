\abstractEN % Do NOT modify this line
This document presents a thesis made on a joint-project named CASCIFFO, between the Lisbon Superior Institute of Engineering - Instituto Superior de Engenharia Lisboa (ISEL), and the Hospital Professor Doutor Fernando Fonseca (HFF). This project was made possible through the funding earned from the Clinical Investigation Agency Award and Biomedical Innovation (AICIB). 
This thesis aims to develop a platform and provide innovative mechanisms for interoperability with internal or external information systems, allowing, when desired, data synchronization, index search, identification data management and even access to detailed clinical data, the ability to manage and monitor clinical trials as well as their participants within the Clinical Research Center. 
The main contribution of this thesis will the optimization and simplicity in managing clinical trials in order to facilitate the researchers efforts in the management of clinical trials.
We believe this platform to be an important step in the modernization of the HFF and UIC, bringing change in how the researchers view their institute and its importance. 
This is report aims to specify the functional requisites of in the platform CASCIFFO and situate the current status of development within the expected timeline. 

% The dissertation must contain two versions of the abstract, one in the same language as the main text, another in a different language.  The package assumes the two languages under consideration are always Portuguese and English.

% The package will sort the abstracts in the proper order. This means the first abstract will be in the same language as the main text, followed by the abstract in the other language, and then followed by the main text. 

% The abstract is critical because many researchers will read only that part. Your abstract should provide an accurate and sufficiently detailed summary of your work so that readers will understand what you did, why you did it, what your findings are, and why your findings are useful and important. 

% The abstract should not contain bibliography citations, tables, charts or diagrams.  Abbreviations should be limited. Abbreviations that are defined in the abstract will need to be defined again at first use in the main text. 

% Finally, you must avoid the use of expressions such as "The present work deals with ... ", "In this thesis are discussed .... ", "The document concludes that .... ", "apparently and .... " etc.

% The word limit should be observed, 300 words is the limit.

% Abstracts are usually followed by a list of keywords selected by the author. Choosing appropriate keywords is important, because these are used for indexing purposes. Well-chosen keywords enable your manuscript to be more easily identified and cited. 

% Keywords of abstract in English
\begin{keywords}
Management of Clinical Trials, Clinical Research, UIC,
\end{keywords} 
