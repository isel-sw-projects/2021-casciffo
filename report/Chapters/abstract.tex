\abstractEN % Do NOT modify this line
This document presents a thesis made on a joint-project named CASCIFFO, between the Lisbon Superior Institute of Engineering - Instituto Superior de Engenharia Lisboa (ISEL), and the Hospital Professor Doutor Fernando Fonseca (HFF). 

This thesis aims to develop a platform and provide innovative mechanisms for interoperability with internal or external information systems, allowing, when desired, data synchronization, index search, access to detailed clinical data, the ability to manage and monitor clinical trials as well as their participants within the Clinical Research Center. 

The platform, is a web app following the \acrfull{spa} architecture and uses a \textit{PostgreSQL} database system to store its data. It is split in two main modules, the \acrfull{be} module and the \acrfull{fe} module. 

The \acrshort{fe} module consists in the client application where use interaction begins and was developed in a \textit{NodeJs} environment utilizing the \textit{ReactJs} framework. 

The \acrshort{be} module was developed as a RESTful API through HTTP. It was implemented using \textit{Spring WebFlux} allowing the communication and environment to be fully non-blocking I/O from the moment a request is received, processed and a response is sent, including any possible database access calls.

The developed solution was tested with unit and integration tests along with a continuous development approach with the use of the \textit{Heroku} cloud service and finally hosted via Apache in an internal server inside the hospital's facilities.

The main contribution of this thesis are the optimization and simplification in managing clinical trials in order to facilitate the researchers efforts in the management of clinical trials.
It is our belief that this platform is a significant step in modernizing the \acrshort{hff} and \acrshort{uic}, changing the researchers perception of the \acrshort{hff} bringing about greater focus on the institute and it's significance.
% The dissertation must contain two versions of the abstract, one in the same language as the main text, another in a different language.  The package assumes the two languages under consideration are always Portuguese and English.

% The package will sort the abstracts in the proper order. This means the first abstract will be in the same language as the main text, followed by the abstract in the other language, and then followed by the main text. 

% The abstract is critical because many researchers will read only that part. Your abstract should provide an accurate and sufficiently detailed summary of your work so that readers will understand what you did, why you did it, what your findings are, and why your findings are useful and important. 

% The abstract should not contain bibliography citations, tables, charts or diagrams.  Abbreviations should be limited. Abbreviations that are defined in the abstract will need to be defined again at first use in the main text. 

% Finally, you must avoid the use of expressions such as "The present work deals with ... ", "In this thesis are discussed .... ", "The document concludes that .... ", "apparently and .... " etc.

% The word limit should be observed, 300 words is the limit.

% Abstracts are usually followed by a list of keywords selected by the author. Choosing appropriate keywords is important, because these are used for indexing purposes. Well-chosen keywords enable your manuscript to be more easily identified and cited. 

% Keywords of abstract in English
\begin{keywords}
Management of Clinical Trials, Clinical Research, UIC, Hospital Professor Doutor Fernando Fonseca, HFF, Reactive, Non-blocking, SPA, React, Apache, Spring, Spring WebFlux, PostgreSQL
\end{keywords} 
