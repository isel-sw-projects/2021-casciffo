% 
%  chapter2.tex
%
\chapter{State of the art}\label{ch:state-of-the-art}

This chapter consists of a detailed view over the concept and functional requirements of the CASCIFFO platform.
It is structured with the following sections:
\begin{itemize}
    \item Related work: Analysis of platforms with similar functionalities.
    \item Infrastructure: Description of technologies and framework used for the development of CASCIFFO. 
    \item Access control: Identification and categorization of actors and their roles.
    \item Processes: Identification and detailing of the process flow.
\end{itemize}

\section{Related work}
Clinical trial management is an important factor to consider for any health care organization, to be able to track and manage any data concerning clinical trials and studies, from the moment of the proposal, to the clinical trial itself, the patient recruitment and monitoring management. To this effect, after an brief search and analysis of platforms providing these features, two platforms stood out: the Clinical Conductor Clinical Trial Management System (CTMS)~\cite{clinical-conductor-ctms} by Advarra~\cite{Advarra} and RealTime-CTMS. Clinical Conductor CTMS is a premium service scalable to optimize finances, regulatory compliance, and overall clinical research operations such as financial, patient and visit management including patient recruitment. The platform RealTime-CTMS is a leader in cloud-based software solutions for the clinical research industry and is dedicated to solving problems and providing systems that make the research process more efficient and more profitable~\cite{realtime-ctms}.
CASCIFFO takes a tailored approach in building a straightforward platform that fits the needs of the Hospital Professor Doutor Fernando Fonseca. It features, in similarity to the aforementioned platforms, clinical trial management, from the moment its a proposal to the end of the clinical trial, allowing the entire progress to be tracked and analyzed; time line management in the aspect of tracking progress and reporting the completion date as well as possible overdue target dates; patient monitoring and visit management; financial management of clinical trials and each individual member of the investigation team and a role-based system for users of different internal departments. Furthermore the way CASCIFFO is planned to be developed, will allow for offline use. It is not as robust as a leading cloud-based software solution like RealTime-CTMS, however, it accomplishes its purpose of bringing innovation and simplicity to the management of clinical trials of the Hospital Professor Doutor Fernando Fonseca's Clinical Research Unit (\acrlong{uic}).

\section{Infrastructure}
\label{sec:infrastructure}
The infrastructure of CASCIFFO, as mentioned previously, consists of two core modules, the front-end and the back-end.  
The front-end runs on a \textit{Node.js} environment, using \href{https://reactjs.org/}{React}, a JavaScript library for building user interfaces,  with Typescript to build all front-end functionalities. The dependencies are managed and installed using \href{https://docs.npmjs.com/about-npm}{npm}, a software package manager, installer and the worlds largest software library. \textit{npm} was chosen over the package manager \href{https://yarnpkg.com/}{yarn} due to its larger community and support.
The back-end executes on a Java virtual environment, utilizing \textit{Spring Boot} and \textit{Spring WebFlux} as the basis for building the server. The \textit{Spring Boot} framework facilitates the building and deployment of web applications by removing much of the boilerplate code and configuration associated with web development. The \textit{Spring WebFlux}~\cite{spring-webflux} framework, despite being a new technology, was chosen for its non-blocking I/O web stack framework. This technology allows for the use of Spring Data R2DBC, a driver that overcomes the inherent blocking I/O limitation of drivers such as JDBC when accessing data in a database. The database connections via the R2DBC driver are non-blocking, returning reactive streams of data in the form of Flux and Mono upon access.
CASCIFFO has its own database, utilizing the framework \textit{PostgreSQL}, a powerful open-source object-relation database system. \textit{PostgreSQL} was chosen for its earned reputation in its proven architecture~\cite{postgresql}, reliability, data integrity and community support. While CASCIFFO has its own database, the patient and medical staff data is planned to be imported from an internal database, "Admission", within the \acrshort{hff}/\acrshort{uic}. There are restrictions on the amount of queries made on the medical staff information, limiting this procedure to once per day.


\section{Access control}
\label{sec:access-control}
Within the app CASCIFFO, in order for the management of clinical investigations to progress, it needs to be reviewed by many entities, such as the Administrative Council ("Concelho administrativo", CA), the Finance and Juridical department.
Given this nature of CASCIFFO, there needs to be a well-defined structure of access control, so that each entity can contribute to the management of clinical investigations within the scope of their responsibilities. Each involved entity must have a role and a set of permissions.  
The roles identified are as follows: the \acrshort{uic} role, given to the investigators who can create and edit clinical investigation proposals; the Team Member role, given to investigators belonging to the team conducting the clinical investigation; the Management role, able to approve and reject clinical investigation proposals; the Finance and Juridical roles, given to collaborators who's function belongs within the Finance and Juridical departments, respectfully; and finally the Superuser role, who has complete access to every feature CASCIFFO has to offer. Each user of CASCIFFO have multiple roles.

\section{Processes} 
\label{sec:processes}
This section details the types of processes occurring within the scope of the project.  
There are three identified processes consisting of the life-cycle of a clinical investigation proposal, the Clinical Trials and the contract addenda. 



\subsection{Clinical Investigation Proposals}
\label{subsec:clinical-investigation-proposals}
From the instant a clinical investigation is kicked-off, it follows through a series of states and protocols that must be adhered to, in order to be completely validated.
There are two types of clinical investigations: Clinical Trials and Observational Trials.  
Each state, except the terminal one, has an entity responsible, `owner`, for advancing the state. 
The flow of states is as follows:  

\begin{enumerate}
    \item Submitted ("Submetido"), `owner=\acrshort{uic}`;
    \item Financial Contract Validation ("Validação do CF"), `owner=Finance, Juridical`;
    \item External validation ("Validação externa"), `owner=\acrshort{uic}`;
    \item Submission to the CA ("Submissão ao CA"), `owner=\acrshort{uic}`;
    \item Internal validation ("Validação interna"), `owner=CA`;
    \item Validated ("Validado").
\end{enumerate}

The enumerated set of states corresponds to the life-cycle of a clinical trial Proposal. An Observational Trial Proposal consists of the enumerated states 1, 4, 5 and 6; it lacks a financial component and a promoter.  

Taking the example of the submission of a clinical trial Proposal, an investigator starts by creating and submitting a proposal. Once it's submitted, the CA will be notified, via app and email. This state is described as \textit{Submitted}.  
When the negotiation of the financial contract begins, the principal Investigator, who belongs to the \acrshort{uic} role, will advance the state to its next step in the proposal's evolution, \textit{Validation of financial contract}, where users with roles of either `Finance` and/or `Juridical`, which represents the Financial and Juridical internal departments, respectfully, will be notified that a proposal is ready to be validated. The validation will consist of a simple `Accept` or `Refuse` with added justification for the choice. In the case of either user with `Finance` or `Juridical` role reject the financial contract, both the validations will be reset and the principal investigator will be notified of the occurrence.  
Once it's accepted by both roles, the proposal will automatically advance into the next state, \textit{External validation}. In this state the \acrshort{uic} will be notified of the change and asked to verify all the documents, including the final version of the financial contract. The \acrshort{uic} to the external promoter, requesting their signatures. When the reply is received via email, the \acrshort{uic} adds the received signatures and possible additional documents to the proposal in the CASCIFFO platform, advancing it to the next stage \textit{Submission to CA}.  
In the state \textit{Submission to CA}, the principal investigator will be notified both via the platform and email that a proposal requires their signature. Once the principal investigator submits their signature into the platform, he can advance the proposal's state into \textit{Internal Validation}. The progression to the mentioned state will notify users with the `CA` role stating that a proposal is ready for its final evaluation.  
Once a user with the role of `CA` checks the proposal he can either validate it or not. In case it's not validated, the proposal will become `canceled` with its life-cycle ending there, however, if it is validated, the proposal can become fully validated once the termination of another process is ends successfully.  
This process, which can be considered a sub-process, is called the validation protocol. It starts in parallel when the proposal is first submitted.  
The purpose of this protocol is to validate the clinical investigation's ethical and safety values. It consists in the validation of the proposal by an internal agency, the \href{https://www.ceic.pt/}{clinical investigations Ethics Comity ("Comissão de Ética para Investigação Clínica", CEIC~\cite{ceic})}. The protocol ends when the mentioned agency approves or rejects the proposal.  
Once it has successfully passed through the described validation protocol, the proposal becomes \textit{Validated} and a Clinical or Observational Trial is automatically created, importing the core information from the proposal.  
If either process declares the proposal invalid, its state becomes `canceled`, notifying the \acrshort{uic} and showing the root cause of cancellation.

Each proposal is distinguished by six main properties, the principal investigator, the type of investigation, the type of therapeutic service it's integrated into (\textit{i.e.} Oncology), the `Sigla` which represents the name of the therapeutic or medicine, the partnerships involved in the investigation and the medical team participating in the investigation.  
Proposals with a financial component must also include the promoter of the investigation, in addition to the properties listed.  

\subsection{Clinical Trials} 
\label{subsec:clinical-trials}
The life-cycle of a clinical trial is divided into three states: active, completed, and canceled.  
Starting with the active state, a clinical trial will become available for viewing and editing once its proposal has been accepted. 
Clinical trials, as a process, consist on the experimentation of new medicine or treatment on a set of participants. These participants can be added to the clinical trial directly from the application. The experimentation requires constant monitoring, through visits, on each participant. These visits can be scheduled either when a participant is added or created afterwards.  
In addition to monitoring participants, several studies can be made in the scope of the clinical trial, such as scientific articles, presentations, reports, etc. 

\subsection{Addenda to the contract}
\label{subsec:addenda-process}
Throughout the life of a clinical trial, there can be made changes to the study's contract, be it changing the investigator team or other factors that impact the standard run of the study. These changes pass through two entities before being applied; the \acrshort{uic} and the CA.  
The addenda can only be made once a clinical trial is active, which means its proposal has already been approved.  
The addenda has five different states: submitted `Submetido`, internal validation by \acrshort{uic} `Validação interna`, internal validation by CA `Validação interna`, the terminal state validated `Validado` and finally the terminal canceled state `Indeferido`.  
The first four mentioned states are sequential, with the last one being an exception state. The sequential flow of states have an entity responsible for advancing their state, the `owner`. Listed below, in similar fashion to the states presented in the proposal process, is the aforementioned sequence:
\begin{enumerate}
    \item Submitted ("Submetido"), `owner=\acrshort{uic}`;
    \item Internal validation ("Validação interna"), `owner=\acrshort{uic}`;
    \item Internal validation ("Validação interna"), `owner=CA`;
    \item Validated ("Validado").
\end{enumerate}


