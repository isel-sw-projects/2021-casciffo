% 
%  chapter1.tex
%
\chapter{Introduction}
\label{ch:intro}
This chapter reviews the introduction of the Hospital Professor Doutor Fernando Fonseca, followed by its Clinical Research Unit and their missions. It also introduces the platform being developed in the scope of this thesis, the motive behind it and its main goals to be achieved. The end of chapter describes the structure of the document.

\section{Context and Motivation}
This thesis is a part of the joint-project CASCIFFO and it's made within the scope of the curricular unit Tese Final de Mestrado, in the course Mestrado em Engenharia Informática e de Computadores. A strong motivator behind this project is the impact it'll have within the Clinical Research Unit, by facilitating the management and monitoring of clinical trials, CASCIFFO aims to alleviate this burden off the workload of researchers.

\section{Hospital Professor Doutor Fernando Fonseca}
\label{ch:intro:sec:intro}

The \href{https://hff.min-saude.pt/}{\acrfull{hff}}, first opened in 1995, is a first of the line hospital combining the professionals excelency with the most modern medical practices, searching to answer to a population of over 600.000 inhabitants of the municipalities of Amadora and Sintra~\cite{hff-intro}. The Institution develops assistance and research activities as well as providing education, pre- and post-graduation training.

\subsection{HFF's mission}
\acrshort{hff}'s mission is to provide humanized and differentiated health
care throughout a person's life cycle, in collaboration with primary and
continuing health care, as well as other hospitals in the National
Health Service's ('Serviço Nacional de Saúde') network~\cite{hff-uic}.

\subsection{Clinicial Research Unit}
The \acrfull{uic}, created in March 2018, is an internal department within the
Hospital that incorporates fundamental concepts of activities in line
with the strategic objectives of the institution. The \acrshort{uic} is responsible
for managing clinical trials and is characterized by a
multidisciplinary team responsible for ensuring accuracy in the
scientific planning of the submitted studies, for the fulfillment of
clinical best practices by researchers and for the negotiation of
contracts for projects financed by external promoters.\\

\subsection{UIC's mission}
\acrshort{uic}'s mission is to promote quality clinical trials in an organized and sustainable manner. It achieves this by following guidelines that value systematic knowledge through the management of interfaces associated with Research, Development, and Innovation. In addition, it also complies with applicable ethical and legal provisions, for the benefit of the Hospital, the Community, the Patients, and the Families/Caregivers.
The goal is to become a reference in the promotion of best practices in hospital clinical research and to consolidate a transversal scientific culture within the institution~\cite{hff-uic}.

\subsection{Clinical Research}
Clinical research is a very important sector to the world of medicine and
health care. It's through it that new medicine and new ways of
treatment are discovered and tested through strict adherence to 
scientific accuracy measurements and best practices before being
administrated to the general public. 

\subsubsection{Types of Clinical Research}
There are two main types of clinical investigations, without intervention, which includes observational trials ("Estudos Observacionais"), and with intervention, which includes clinical trials ("Ensaios Clínicos"). The distinction between these trials comes down to the type of intervention between the study and the participants. Active intervention occurs when the researchers in a trial introduce any variable, such as a new medicine, that provokes any sort of change within the participant's behavior, health care or mindset. No intervention means that the team of researchers will not intervene in any way with the participants besides only monitoring them.
Having stated these differences, we can clearly define observational trials and clinical trials.
Observational Trials have no active intervention, they instead contemplate purely the aspect of observation, for example in the evaluation of a potential risk factor. On the other hand, Clinical Trials engage in active intervention, as such they can be characterized as a Clinical Research which involves any intervention that foresees any change, influence or programming of health care, behavior or knowledge of the participating patients or caretakers, with the end goal of discovering the effects it had on the participant's health.
Clinical trials consist of a scientific controlled investigation, done
on humans (healthy or ill), with the end-goal of establishing or confirming the safety and efficiency of the experimental medicine~\cite{tipos-de-ensaios}.  

Clinical trials have shown to be a vital tool in the development and testing of vaccinations and treatments for the safety of the entire world population, especially now during the present \textit{Novel Coronavirus (SARS-CoV-2)} pandemic. For this reason, the efficiency and simplicity in managing and monitoring the evolution of a clinical trial is essential.  

\section{Project overview and main goals}\label{ch:intro:sec:casciffo}

CASCIFFO is a joint project between \href{https://hff.min-saude.pt/}{\acrshort{hff}} and \href{https://www.isel.pt/}{ISEL} and was developed through funding by the Clinical Investigation Agency Award and Biomedical Innovation (AICIB).  
CASCIFFO is a web-app that aligns with the \acrshort{uic}'s goals by promoting efficiency and quality in the management of clinical research.
CASCIFFO strives to make the visualization, monitoring, and management of clinical Trials as simple and straightforward as possible.   
It will allow the UIC/HFF to be modernized, bringing a shift in how patients, researchers, and promoters view and value their institution.  

\subsection{Challenges}
The current procedure of a clinical investigation relies on e-mail exchanges between the external and internal parties involved, which adds a considerable amount of effort in the management and monitoring of clinical trials. Another concern is with the scheduling and monitoring of Clinical Trials patients, as there is currently no systematic way of distinguishing the types of appointments made for each patient.  
In this context, the application CASCIFFO, aims to provide a solution in order to enhance the efficiency in the management and monitoring of clinical research.

\subsection{Main Goals}
The application aims to develop and provide innovative mechanisms for interoperability with internal and external information systems, allowing, when desired, data synchronization, index search, identification data management and even access to detailed clinical data. 
CASCIFFO consists of two core modules, the \acrfull{fe} module and the \acrfull{be} module. The \acrshort{fe} supports interaction with users while the \acrshort{be} connects to an internal database system to the \acrshort{hff}/\acrshort{uic} and which aggregates the total information of this ecosystem.  
The interaction with users will depend on their role within the platform, displaying the appropriate information to each one. 


\section{Document Structure}
This document is divided into the following 6 chapters: 

\begin{itemize}

    \item Chapter~\ref{ch:intro}, this one, consists of the introduction of the \acrshort{hff} and \acrshort{hff}, their missions and goals, followed by explaining what a clinical research is and their types and concluding with an overview of the project and the goals to achieve. 

    \item Chapter~\ref{ch:state-of-the-art} describes the state of art of CASCIFFO and other related work.

    \item Chapter~\ref{ch:functionalities} consists of the business logic rules of CASCIFFO and the functional requirements. 

    \item Chapter~\ref{ch:architecture} presents the architecture and the infrastructure of the platform, as well as the justifications for the choices made. 

    \item Chapter~\ref{ch:impl} describes the implementation details of both modules that make up the platform. 

    \item Chapter~\ref{ch:eval} describes evaluation of the developed solution.

    \item Lastly chapter~\ref{ch:conclusion} consists of the conclusions and possible next steps in the scope of the platform.

\end{itemize}
