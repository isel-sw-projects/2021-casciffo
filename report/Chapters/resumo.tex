\abstractPT  % Do NOT modify this line

Com o intuito de modernizar o software utilizado pela Unidade de Investigação Clínica do Hospital Professor Doutor Fernando Fonseca, o projeto CASCIFFO foi realizado em parceria entre o Instituto Superior de Engenharia de Lisboa (ISEL) e o Hospital Professor Doutor Fernando Fonseca. O financiamento do projeto foi possibilitado pela atribuição do prémio pela Agência de Investigação Clínica
e Inovação Biomédica (AICIB).
O projeto tem como objetivos desenvolver uma plataforma, CASCIFFO, e fornecer mecanismos inovativos na interoperabilidade com sistemas de informação externos ou internos ao hospital. Permitir, quando desejado, sincronização de dados, índices de pesquisa, identificação de gestão de dados e acesso a detalhes de ensaios clínicos. Tem ainda como objetivo permitir a gestão e monitorização de ensaios clínicos, incluindo a gestão de participantes nos ensaios.

A contribuição principal esperada por este projeto será a otimização e simplicidade na gestão de ensaios clínicos no âmbito de facilitar o esforço feito pelos investigadores nesta gestão.
Deste modo, espera-se que esta modernização venha a trazer uma mudança no pensamento dos investigadores na sua visão da instituição e a sua importância.


% Independentemente da língua em que está escrita a dissertação, é necessário um resumo na língua do texto principal e um resumo noutra língua.  Assume-se que as duas línguas em questão serão sempre o Português e o Inglês.

% O \emph{template} colocará automaticamente em primeiro lugar o resumo na língua do texto principal e depois o resumo na outra língua. 

% Resumo é a versão precisa, sintética e selectiva do texto do documento, destacando os elementos de maior importância. O resumo possibilita a maior divulgação da tese e sua indexação em bases de dados.

% A redação deve ser feita com frases curtas e objectivas, organizadas de acordo com a estrutura do trabalho, dando destaque a cada uma das partes abordadas, assim apresentadas: Introdução; Objectivo; Métodos ; Resultados  e Conclusões

% O resumo não deve conter citações bibliográficas, tabelas, quadros, esquemas. 

% E, deve-se evitar o uso de expressões como "O presente trabalho trata ...", "Nesta tese são discutidos....", "O documento conclui que....", "aparentemente é...." etc. 

% Existe um limite de palavras, 300 palavras é o limite.

% Para indexação da tese nas bases de dados e catálogos de bibliotecas devem ser apontados pelo autor as palavras-chave que identifiquem os assuntos nela tratados. Estes permitirão a recuperação da tese quando da busca da literatura publicada. 

% Keywords of abstract in Portuguese
\begin{keywords}
Gestão de ensaios clínicos, Investigação Clínica, Unidade de Investigação Clínica, Hospital Professor Doutor Fernando Fonseca, HFF, Reativo, Não-bloqueante
\end{keywords} 
% to add an extra black line
