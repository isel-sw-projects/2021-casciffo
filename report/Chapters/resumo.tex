\abstractPT  % Do NOT modify this line

Com o intuito de modernizar o software utilizado pela Unidade de Investigação Clínica do Hospital Professor Doutor Fernando Fonseca, o projeto CASCIFFO foi realizado em parceria entre o Instituto Superior de Engenharia de Lisboa (ISEL) e o Hospital Professor Doutor Fernando Fonseca. 

O projeto tem como objetivos desenvolver uma plataforma, CASCIFFO, que fornece mecanismos inovativos na interoperabilidade com sistemas de informação externos ou internos ao hospital. Estes mecanismos incluem permitir, quando desejado, sincronização de dados, índices de pesquisa, acesso a detalhes de ensaios clínicos, a gestão e monitorização de ensaios clínicos, incluindo os seus participantes.

A plataforma desenvolvida segue a arquitetura \acrfull{spa} e utiliza a sua própria base de dados criada com o sistema de base de dados \textit{PostgreSQL}. Esta plataforma é constituída por dois módulos essenciais, o \acrfull{fe} e o \acrfull{be}.

O módulo \acrshort{fe} representa a aplicação do lado do cliente onde a interação com o utilizador começa. Este módulo foi desenvolvido num ambiente \textit{NodeJs} e utiliza a infraestrutura \textit{ReactJs}.

O módulo \acrshort{be} foi desenvolvido com o objetivo de ser um servidor que dispõe de um serviço REST API que permite comunicação através de HTTP. Este módulo utiliza a infraestrutra \textit{Spring WebFlux} o que favorece um ambiente e comunicação \textit{non-blocking I/O} do momento que um pedido HTTP é recebido, processado e gerado a resposta. Este processamento incluí pedidos de acesso à base de dados de forma assíncrona.

A solução desenvolvida foi testada com testes unitários e de integração em conjunto de um ambiente que possibilita a metodologia de desenvolimento contínuo ao utilizar o serviço \textit{cloud} \textit{Heroku}. 
Uma vez testada, a plataforma foi alojada num server interno no hospital e servida pelo serviço \textit{Apache}.

A contribuição principal esperada por este projeto será a otimização e simplicidade na gestão de ensaios clínicos no âmbito de facilitar o esforço feito pelos investigadores nesta gestão.
Deste modo, espera-se que esta modernização venha a trazer uma mudança no pensamento dos investigadores na sua visão da instituição e a sua importância.
% Independentemente da língua em que está escrita a dissertação, é necessário um resumo na língua do texto principal e um resumo noutra língua.  Assume-se que as duas línguas em questão serão sempre o Português e o Inglês.

% O \emph{template} colocará automaticamente em primeiro lugar o resumo na língua do texto principal e depois o resumo na outra língua. 

% Resumo é a versão precisa, sintética e selectiva do texto do documento, destacando os elementos de maior importância. O resumo possibilita a maior divulgação da tese e sua indexação em bases de dados.

% A redação deve ser feita com frases curtas e objectivas, organizadas de acordo com a estrutura do trabalho, dando destaque a cada uma das partes abordadas, assim apresentadas: Introdução; Objectivo; Métodos ; Resultados  e Conclusões

% O resumo não deve conter citações bibliográficas, tabelas, quadros, esquemas. 

% E, deve-se evitar o uso de expressões como "O presente trabalho trata ...", "Nesta tese são discutidos....", "O documento conclui que....", "aparentemente é...." etc. 

% Existe um limite de palavras, 300 palavras é o limite.

% Para indexação da tese nas bases de dados e catálogos de bibliotecas devem ser apontados pelo autor as palavras-chave que identifiquem os assuntos nela tratados. Estes permitirão a recuperação da tese quando da busca da literatura publicada. 

% Keywords of abstract in Portuguese
\begin{keywords}
Gestão de ensaios clínicos, Investigação Clínica, Unidade de Investigação Clínica, Hospital Professor Doutor Fernando Fonseca, HFF, Reativo, Não-bloqueante, SPA, React, Apache, Spring, Spring WebFlux, PostgreSQL
\end{keywords} 
% to add an extra black line
